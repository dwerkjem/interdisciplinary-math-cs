\chapter{Preface}
\section{Preface}\label{sec:preface}

\subsection{Abbreviations}

The following abbreviations are used throughout this paper:

\subsubsection{Computer Science Abbreviations}

\begin{tabular}{ll}
    \textbf{CS} \& Computer Science \\
    \textbf{DS} \& Data Structures \\
    \textbf{OOP} \& Object-Oriented Programming \\
    \textbf{AI} \& Artificial Intelligence \\
    \textbf{DBMS} \& Database Management System \\
    \textbf{ML} \& Machine Learning \\
    \textbf{OS} \& Operating System \\
    \textbf{API} \& Application Programming Interface \\
\end{tabular}

\subsubsection{Mathematics Abbreviations}

\begin{tabular}{ll}
    \textbf{DM} \& Discrete Mathematics \\
    \textbf{P \& C} \& Probability and Combinatorics \\
    \textbf{S \& L} \& Sets and Logic \\
    \textbf{T \& F} \& Truth and Falsity \\
    \textbf{FFT} \& Fast Fourier Transform \\
    \textbf{LCM} \& Least Common Multiple \\
    \textbf{GCD} \& Greatest Common Divisor \\
    \textbf{PDF} \& Probability Density Function \\
\end{tabular}

\subsubsection{General Abbreviations}

\begin{tabular}{ll}
    \textbf{STEM} \& Science, Technology, Engineering, and Mathematics \\
    \textbf{UI} \& User Interface \\
    \textbf{UX} \& User Experience \\
    \textbf{BFS} \& Breadth-First Search \\
    \textbf{DFS} \& Depth-First Search \\
    \textbf{RSA} \& Rivest-Shamir-Adleman (encryption algorithm) \\
\end{tabular}

\subsection{Figures} \listoffigures % Correct LaTeX command to print the list of figures

\subsection{Tables} \listoftables

\subsection{Code} \lstlistoflistings

\subsection{Introduction}

This paper is a collection of thoughts and ideas that have been developed over the past few years. The concepts presented here are the result of my experiences as a computer science student. Although I have always been interested in mathematics, it wasn't until I approached the subject from a computer science perspective that I fully appreciated its beauty. This paper is an effort to share some of the insights I have gained through this interdisciplinary lens.

\subsection{Acknowledgements}

This work was completed independently, but I am deeply grateful to Weber State University for providing the resources and opportunities that have enabled me to pursue my interests. I would also like to thank my family and friends for their support and encouragement. Finally, I extend my gratitude to the authors of the textbooks and papers I have referenced throughout this work. Their contributions have been invaluable in helping me develop my own ideas.

\subsection{Abstract}

This paper explores the relationship between mathematics and computer science, focusing on the ways in which that students can benefit from an interdisciplinary approach to these fields. The paper begins by comparing elementary mathematical concepts to their computer science counterparts and beginner programming concepts to their mathematical equivalents. It then discusses the benefits of teaching mathematics and computer science together, including the development of problem-solving skills, the promotion of creativity, and the enhancement of students' understanding of both subjects. As the paper progresses, it delves into more advanced topics, such as the connections between discrete mathematics and computer science, the role of algorithms in both fields, and the ways in which computer science can be used to solve complex mathematical problems. The paper concludes by emphasizing the importance of interdisciplinary education and encouraging educators to incorporate elements of both mathematics and computer science into their curricula. The goal of this paper is to inspire students and educators alike to explore the connections between these two disciplines and to appreciate the beauty and power of mathematics and computer science when studied together.

